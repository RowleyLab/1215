\documentclass[12pt, letterpaper]{article}
\usepackage{SyllabusStyle}

\begin{document}
\begin{center}
	{\Large \textsc{Principles of Chemistry Lab I}}

	CHEM 1215
\end{center}
\begin{center}
	{\large Fall 2023}
\end{center}
\begin{center}
	\rule{0.99\textwidth}{0.4pt}
	\begin{tabular}{llcll}
		\textbf{Instructor:} & Matthew Rowley           &  & \textbf{Office Hours:} & MWRF 10:00 am -- 11:00 am \\
		\textbf{Telephone:}  & (435) 586-7875           &  &                        & T 1:00 pm -- 2:00 pm    \\
		\textbf{Email:}      & matthewrowley$1$@suu.edu &  & \textbf{Office:}       & SC-220                   \\
		\multicolumn{5}{c}{Please include the course number in the subject line of all correspondence.}
	\end{tabular}
	\rule{0.99\textwidth}{0.4pt}
\end{center}

\section*{Course Description}
This is the lab to accompany CHEM 1210. A mimimum grade of ``C'' (2.0 or above) must be earned in this course before it can be counted toward a physical science major or minor or as a prerequisite for any other course.

\paragraph{Prerequisites:}
None

\paragraph{Concurrent requisite:}
CHEM 1210 -- Principles of Chemistry I

\paragraph{Course Materials:} ~

\emph{CHEM 1215 Experiments for Chemical Principles I} by Bronsema (Available at the SUU Bookstore)

You are required to bring and wear your own pair of OSHA-approved safety goggles to \emph{every} lab. Students without eye protection will be required to leave the lab and will receive a zero for the labwork that day.

\paragraph{Student Learning Outcomes:}
\begin{description}
	\item[Knowledge of the physical and natural world] -- Students will recall, interpret, compare, explain, and apply chemistry terminology and theory.
	\item[Quantitative Literacy] -- Students will use chemical equations, graphs and tables to interpret and communicate chemical information.
	\item[Inquiry and Analysis] -- Students will investigate chemical problems.
	\item[Communication] -- Students will report laboratory results clearly and concisely.
	\item[Problem Solving] -- Students will implement experimental procedures.
	\item[Teamwork] -- Students will productively interact with each other to successfully conduct chemistry experiments.
\end{description}

\section*{Laboratory Work}
Before lab, you are expected to have read the handout of your experiment as well as review your lecture notes from class. Come prepared to enter your data into the lab computers and have a USB drive with you. You may perform each laboratory with a lab partner and you may acquire your data together during your scheduled lab time. However, you must NOT work with your lab partner beyond this. All analysis of data and calculations as well as all laboratory reports must be done on an individual basis. Failure to do so will result in a zero for the lab in question.

\noindent Please follow all safety procedures, especially by wearing safety glasses or goggles. When leaving the lab, please make sure it is in the same condition as it was when you arrived. Be respectful of others.

\paragraph{Laboratory Risk:}
Chemical exposure is a constant risk in a chemistry lab. To minimize the risk to yourself and those around you, the following rules must be followed:
\begin{itemize}
	\item Never taste or smell a chemical or pipette by mouth.
	\item Wash your hands before leaving the lab and frequently during the lab to avoid accidental contamination of yourself and others.
	\item Dispose of chemicals only as directed. Nothing goes down the sink unless expressly directed.
	\item Keep your work area clean; wipe up any spills (liquid or solid) immediately.
	\item Replace caps on reagent bottles, and never return chemicals to the original container.
	\item No shorts, tank tops, or sandals allowed in lab, and long hair should be restrained.
	\item Wear safety glasses at all times when in the lab.
\end{itemize}
Students enrolling in this course should realize that they are voluntarily exposing themselves to a variety of chemicals, some of which could be irritating or hazardous with excessive exposure.  For those persons with a sensitive medical condition such as allergies, precautions such as wearing additional protective garments, delaying enrolling, or even not enrolling in a class may be necessary.

\section*{Tentative Schedule}
 We will meet in room 208 of the Skaggs Center for Health \& Molecular Sciences (SCA)
\begin{description}
	\item Section 03 will meet on Tuesdays from 8:00-10:50
	\item Section 06 will meet on Thursdays from 11:00-1:50
\end{description}

\paragraph*{Week 1: Aug 28 -- Sept. 1}~

\textbf{No lab this week}

\paragraph{Week 2: Sept 4 -- Sept. 8}~

Check-in and Safety in the Laboratory

\paragraph{Week 3: Sept. 11 -- Sept. 15}~

Measurement

\paragraph{Week 4: Sept. 18 -- Sept. 22}~

Nomenclature and Formulas

(Note that the nomenclature lab is one you can do at home, but formulas will be done in the lab)

\paragraph{Week 5: Sept. 25 -- Sept. 29}~

Hydrates

\paragraph{Week 6: Oct. 2 -- Oct. 6}~

Limiting Reactant and \% Copper

\paragraph{Week 7: Oct. 9 -- Oct. 13}~

Spectrophotometry

\paragraph{Week 8: Oct. 16 -- Oct. 20}~

\textbf{No lab this week (Fall Break)}

\paragraph{Week 9: Oct. 23 -- Oct. 27}~

Calorimetry

\paragraph{Week 10: Oct. 30 -- Nov. 3}~

Gas Laws

\paragraph{Week 11: Nov. 6 -- Nov. 10}~

Spectroscopy

\paragraph{Week 12: Nov. 13 -- Nov. 17}~

Chemical Reactions of Metals

\paragraph{Week 13: Nov. 20 -- Nov. 24}~

\textbf{No lab this week (Thanksgiving Break)}

\paragraph{Week 14: Nov. 27 -- Dec. 1}~

Reactivity of Halogens and Molecular Geometry

(Note that the geometry lab is one you can do at home, but halogens will be done in the lab)

\paragraph{Week 15: Dec. 4 -- Dec. 8}~

Lab Checkout and Final Exam -- Come as normally scheduled for administration of the final exam. Bring a calculator and any writing utensil (or two).

\paragraph{Finals Week}~

No Final -- You took it last week!

\section*{Course Requirements}
Grades will be based on the following items:
\begin{description}
	\item[Pre-Lab Quizzes] 10 Points Each
	\item[Safety and Clean-up] 5 Points Each
	\item[Lab Reports] 30 Points Each
	\item[Final Exam] 200 Points
\end{description}
Final Grades will be assigned according to the following scale:

\begin{tabular}{rl|c|rl}
	Percentage & Grade &  & Percentage & Grade \\ \midrule
	93.0-100   & A     &  & 73.0-77.0  & C     \\
	90.0-93.0  & A-    &  & 70.0-73.0  & C-    \\
	87.0-90.0  & B+    &  & 67.0-70.0  & D+    \\
	83.0-87.0  & B     &  & 63.0-67.0  & D     \\
	80.0-83.0  & B-    &  & 60.0-63.0  & D-    \\
	77.0-80.0  & C+    &  & < 60.0     & F
\end{tabular}
\paragraph{Pre-Lab Quizzes:}
These quizzes must be completed before the start of lab each week, and will be collected before the lab begins.

\paragraph{Safety and Clean-up:}
Any student who violates laboratory rules or engages in unsafe behavior in the laboratory may lose points. Any student who leaves their station without fully cleaning up after the lab period will likewise lose points.

\paragraph{Lab Reports:}
Lab report pages are included at the end of the instructional material for each lab. These reports are due at the beginning of the next scheduled laboratory day.

\paragraph{Final Exam:}
The final exam is comprehensive and questions will draw on chemical concepts, laboratory techniques, results, and analysis and interpretation.

\paragraph{Attendance Policy:}
Students are expected to attend class. If you must miss class, contact the instructor ahead of time.

\paragraph{Late Work Policy:}
All pre-lab quizzes must be completed before the start of each lab period, and all reports are to be turned in at the \emph{beginning} of the lab period. Late work will not be accepted.

\paragraph{Make-up Work Policy:}
In general, there will be no opportunity to make up missed work. If you must miss class, please contact the instructor ahead of time.

\section*{Miscellany}

\paragraph{Scientific Calculator:}
There are many different ways to calculate figures during homework. It is tempting to rely on Online resources such as \href{http://www.wolframalpha.com}{http://www.wolframalpha.com}, or to simply use a calculator application on a smart phone. During exams, however, any devices capable of connecting to the Internet will \emph{not} be allowed. You will instead need a scientific calculator capable of performing exponentiation and logarithms for the exams. Using this calculator exclusively while doing homework will ensure that you are familiar with it for use during exams.

\paragraph{Important syllabus statements related to ATTENDANCE and COVID-19:} ~

\noindent\emph{What should I expect in the classroom this semester?}

\noindent The following are general guidelines for the classroom environment
\begin{description}
	\item[Class Attendance is Required:] If you are registered for a Face-to-Face, Synchronous Remote, or Hybrid course, attendance is required. If you are ill or instructed to isolate or quarantine, you may request a faculty member record the class and share it with you or you may request other reasonable accommodations. Your instructor will work with you to develop a plan for completing coursework while you are isolated/quarantined. In order for you to receive academic accommodations and ensure that your request is communicated to faculty, you must submit this \href{https://my.suu.edu/covid/selfreport/}{self report form}.
	\item[\href{https://www.suu.edu/registrar/onlinehybrid.html}{Course ~delivery ~modalities} ~are ~posted ~online ~for ~each ~course, ~but ~may ~be ~modified ~in] \textbf{response to emerging COVID conditions:} SUU is employing every effort to maintain a learning environment that is engaging and safe. The course modality listed when you registered for courses should remain for the semester; however, due to COVID conditions, the delivery of modality for a specific course may change during the semester. Normally, these changes will be short term (possibly the length of a quarantine or isolation time period), or in some cases longer. When such a modification is needed, faculty members will work with their department chair and/or dean and the students to maintain an effective learning environment.
\end{description}

\paragraph{Academic Integrity:}
Scholastic dishonesty will not be tolerated and will be prosecuted to the fullest extent (see \href{https://www.suu.edu/policies/06/33.html}{SUU Policy 6.33}). You are expected to have read and understood the current SUU student conduct code (\href{https://www.suu.edu/policies/11/02.html}{SUU Policy 11.2}) regarding student responsibilities and rights, the intellectual property policy (\href{https://www.suu.edu/policies/05/52.html}{SUU Policy 5.52}), information about procedures, and what constitutes acceptable behavior.

\noindent
\underline{Please Note}: The use of websites or services that sell or generate essays (including Artificial Intelligence) is a violation of these policies; likewise, the use of websites or services that provide answers to assignments, quizzes, or tests is also a violation of these policies.

\paragraph{ADA Statement:}
Students with medical, psychological, learning, or other disabilities desiring academic adjustments, accommodations, or auxiliary aids will need to contact the \href{https://www.suu.edu/disabilityservices/}{Disability Resource Center}, located in Room 206F of the Sharwan Smith Center or by phone at (435) 865-8042. The Disability Resource Center determines eligibility for and authorizes the provision of services.

\noindent
If your instructor requires attendance, you may need to seek an ADA accommodation to request an exception to this attendance policy. Please contact the Disability Resource Center to determine what, if any, ADA accommodations are reasonable and appropriate.

\paragraph{Emergency Management Statement:}
In case of emergency, the university's Emergency Notification System (ENS) will be activated. Students are encouraged to maintain updated contact information using the link on the homepage of the \emph{mySUU} portal. In addition, students are encouraged to familiarize themselves with the Emergency Response Protocols posted in each classroom. Detailed information about the university's emergency management plan can be found at: \href{http://www.suu.edu/emergency}{http://www.suu.edu/emergency}

\paragraph{Academic Credit:}
According to the federal definition of a Carnegie credit hour: A credit hour of work is the equivalent of approximately 60 minutes of class time or independent study work. A minimum of 45 hours of work by each student is required for each unit of credit. Credit is earned only when course requirements are met. One (1) credit hour is equivalent to 15 contact hours of lecture, discussion, testing, evaluation, or seminar, as well as 30 hours of student homework. An equivalent amount of work is expected for laboratory work, internships, practica, studio, and other academic work leading to the awarding of credit hours. Credit granted for individual courses, labs, or studio classes range from 0.5 to 15 credit hours per semester.

\paragraph{Non-Discrimination Statement:}
SUU is committed to fostering an inclusive community of lifelong learners and believes our university's encompassing of different views, beliefs, and identities makes us stronger, more innovative, and better prepared for the global society. 

\noindent
SUU does not discriminate on the basis of race, religion, color, national origin, citizenship, sex (including sex discrimination and sexual harassment), sexual orientation, gender identity, age, ancestry, disability status, pregnancy, pregnancy-related conditions, genetic information, military status, veteran status, or other bases protected by applicable law in employment, treatment, admission, access to educational programs and activities, or other University benefits or services.

\noindent
SUU strives to cultivate a campus environment that encourages freedom of expression from diverse viewpoints. We encourage all to dialogue within a spirit of respect, civility, and decency. 

\noindent
For additional information on non-discrimination, please see \href{https://www.suu.edu/policies/05/27.html}{Policy 5.27} and/or visit:\newline \href{https://www.suu.edu/nondiscrimination.}{https://www.suu.edu/nondiscrimination.}

\paragraph{HEOA Compliance Statement:}
For a full set of Higher Education Opportunity Act (HEOA) compliance statements, please visit \href{https://www.suu.edu/heoa}{https://www.suu.edu/heoa}. The sharing of copyrighted material through peer-to-peer (P2P) file sharing, except as provided under U.S. copyright law, is prohibited by law; additional information can be found at \newline\href{https://my.suu.edu/help/article/1096/heoa-compliance-plan}{https://my.suu.edu/help/article/1096/heoa-compliance-plan}.

\noindent
You are also expected to comply with policies regarding intellectual property (\href{https://www.suu.edu/policies/05/52.html}{SUU Policy 5.52}) and copyright (\href{https://www.suu.edu/policies/05/54.html}{SUU Policy 5.54}).

\paragraph{SUUSA Statement:}

As a student at SUU, you have representation from the SUU Student Association (SUUSA) which advocates for student interests and helps work as a liaison between the students and the university administration. You can submit MySUU Voice feedback by going to \href{https://www.suu.edu/suusa/voice}{https://www.suu.edu/suusa/voice}. Likewise, you can learn more about SUUSA’s Executive Council at \href{https://www.suu.edu/suusa/executive-council}{https://www.suu.edu/suusa/executive-council} and about all of SUUSA’s Student Senators at \href{https://www.suu.edu/suusa/senate}{https://www.suu.edu/suusa/senate}. If you have any specific concerns regarding any of your courses, please contact the SUUSA VP of Academics at: \href{suusa_academicsvp@suu.edu}{suusa\_\ignorespaces academicsvp@suu.edu}.

\paragraph{Land Acknowledgement Statement:}
SUU wishes to acknowledge and honor the Indigenous communities of this region as original possessors, stewards, and inhabitants of this Too’veep (land), and recognize that the University is situated on the traditional homelands of the Nung’wu (Southern Paiute People). We recognize that these lands have deeply rooted spiritual, cultural, and historical significance to the Southern Paiutes. We offer gratitude for the land itself, for the collaborative and resilient nature of the Southern Paiute people, and for the continuous opportunity to study, learn, work, and build community on their homelands here today. Consistent with the University's ongoing commitment to equity, diversity, and inclusion, SUU works towards building meaningful relationships with Native Nations and Indigenous communities through academic pursuits, partnerships, historical recognitions, community service, and student success efforts.

\paragraph{Thriving Thunderbirds:}
If you find yourself struggling with mental health issues, please visit \href{https://www.suu.edu/mentalhealth}{https://www.suu.edu/mentalhealth} for access to valuable resources. 

\noindent
Mental health is essential for your academic success. SUU provides resources, support, and services to address mental health issues at every level of concern. We are committed to helping all \href{https://www.suu.edu/mentalhealth/}{Thunderbirds Thrive}. 

\noindent
If you need assistance navigating any of the resources, please contact \href{https://www.suu.edu/caps/}{Counseling and Psychological Services}, the \href{https://www.suu.edu/deanofstudents/}{Dean of Students’ Office}, or the \href{https://www.suu.edu/health/}{Health and Wellness Center}.

\paragraph{Disclaimer:}
Information contained in this syllabus, other than the grading, late assignments, make up work and attendance policies, may be subject to change as deemed appropriate by the instructor.
\end{document}
